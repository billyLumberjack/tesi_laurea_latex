\thispagestyle{empty}
\begin{center}
    {\chapter*{Abstract}} % senza numerazione
\end{center}
\label{abstract}

\addcontentsline{toc}{chapter}{Abstract} % da aggiungere comunque all'indice
The idea of this thesis was born during my internship in Cardioline Spa, a developer and producer Heart Activity technological products company based in Trento.\\While, cloud-based technologies and software interoperability are powerful and interesting tools in order to improve the IT offer and keep up with the times, the Italian Digital Agenda included a strict plan for the public health digitalization.\\Currently, the Clinical Document Repository has just seen a first release, in December 2017.\\Cardioline is already selling a software support to its customers, supporting different kinds of exams: the company provides hospitals, clinics and various stakeholders with Hearth Activity Monitoring machines and the relative WebApplication on their LAN. The approach is limited by the private Network boundaries, the server availability and resources, in favor of an high control, security level and compatibility with legacy applications.\\The goal is to reengineer the softwares, from one-tier to multi-tier, from a stand-alone to a cloud-based infrastructure keeping the same security level and taking into account the laws regulating sensitive data.\\My work in particular was focused on managing EHR (Electronic Health Record) documents, formats, standards, files and communication interfaces to the Italian Clinical Document Repository (FSE - Fascicolo Sanitario Elettronico).\\Furthermore, I dealt with REST-Services (REpresentational State Transfer) in order to manage the new infrastructure, involving third-parties software on the client side.\\A prototype solution has been built to show the possible immediate innovation that can be achieved, suitable for any other company managing Electronic Health Data, which requires to be always available, from any place, with immediate application, and easy to handle in an increasingly smart way.