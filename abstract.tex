\thispagestyle{empty}
\vspace*{\fill}
\begin{center}
    {
    \chapter*{Abstract} % senza numerazione
    }
\end{center}    
    \label{abstract}
    


    \addcontentsline{toc}{chapter}{Abstract} % da aggiungere comunque all'indice
    The work of this thesis was born during my internship in Cardioline Spa, a developer and producer Heart Activity technological products company based in Trento.\\While, cloud-based technologies and software interoperability are powerful and interesting tools in order to improve the IT offer and keep up with the times.\\The Italian Digital Agenda included a strict plan for the public health digitalization, currently, the Clinical Document Repository has just see a first release.\\Cardioline is already selling a software support to its customers, among different kinds of exams: the company provides hospitals, clinics and various stakeholders with Hearth Activity Monitoring machines and the relative WebApplication on their LAN, the approach is limited by the private Network boundaries, the server avaiability and resources, in favor of an high control, security level and compatibility with legacy applications.\\The goal has been to reengineer the softwares, from one-tier to multi-tier, and from a stand-alone to a cloud based infrastructure.\\My work in particular was focused on managing EHR (Electronic Health Record) documents, formats, standards, files and communication interfaces to the Italian Clinical Document Repository (FSE).\\Furthermore I dealt with REST-WebServices so to manage the new infrastructure, involving third-party software on the client side.\\A prototype solution has been built to show the possible immediate innovation that can be made, suitable from any other actor managing Electronic Health Data, which requires to be always available, from anywhere, ready to interact with and easy to handle in an increasingly smart way.
\vspace*{\fill}