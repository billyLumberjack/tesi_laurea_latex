\thispagestyle{empty}
\begin{center}
    {\chapter*{Abstract}} % senza numerazione
\end{center}
\label{abstract}

\addcontentsline{toc}{chapter}{Abstract} % da aggiungere comunque all'indice
The idea of this thesis was born during my internship in Cardioline Spa, a developer and producer Heart Activity technological products company based in Trento.\\Cloud-based technologies and software interoperability are powerful and interesting tools in order to improve the IT offer and keep up with the times for companies. The Italian Digital Agenda included a strict plan for the public health digitalization which implies the development of a Clinical Document Repository. Actually the italian version has just seen a first release, in December 2017.\\Cardioline is already selling a software support to its customers, enabling different kinds of electrocardiography exams: the company provides hospitals, clinics and various stakeholders with a set of Hearth Activity Monitoring machines and an ad hoc WebApplication on their LAN. The current approach is limited by the private Network boundaries, the server availability and resources, in favor of an high control, security level and compatibility with legacy applications.\\The goal is to reengineer the softwares, from one-tier to multi-tier, from a stand-alone to a cloud-based infrastructure keeping the same security level and taking into account the laws regulating sensitive data.\\ The challenge starts by making a realistic proposal solution for Cardioline by exploring the current market offer. Later it will be to develop the proposed solution by respecting each of the constraints.\\ \\My work in particular was focused on managing EHR (Electronic Health Record) documents, formats, standards, files and communication interfaces to the Italian Clinical Document Repository (FSE - Fascicolo Sanitario Elettronico).\\Furthermore, I dealt with REST-Services (REpresentational State Transfer) in order to manage the new infrastructure, involving third-parties software on the client side.\\A prototype solution has been built to show the possible immediate innovation that can be achieved, suitable for any other company managing Electronic Health Data, which requires to be always available, from any place, with immediate application, and easy to handle in an increasingly smart way.\\ \\
The thesis is divided in three main parts. The first one gives the reader on overview of the current state of the art for the technologies being used. A section is dedicated to cloud computing and how is it available to customers and developers. Section 1.2 explains how different systems communicate each other via different web service implementation. The end of the first part, section 1.3, describes briefly what is an electrocardiogram and how is the exam performed until its digitalization and storage into approved national repositories. The second part, is more focused on the specific problem faced, in the first part the analysis and system requirement are explicited while in the last part there's a detailed explanation of the developed prototype.\\The third part concludes this work by outlining the carried innovation and ideas for further improvements.