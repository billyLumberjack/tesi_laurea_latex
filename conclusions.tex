\chapter{Conclusions}
\section{Produced Improvement}
Even in a complex field as e-health, because of sensible data, different country legislations and lots of specific standards each one with pro and cons coming from past years of un-coordinated development, the current technological landscape is slowly converging among formats, interfaces and protocols, thanks to the successful spread of distributed systems and web services.\\
Starting from proprietary softwares each one participating to an e-health system, as the Cardioline one described in \ref{chapter:cardioline_specific_ecosystem}, bounded to each customer company or organization and ending up with a cloud based system (see chapter \ref{chapter:implemented_solution}) which strongly uses international standard to store data and communicate with external services and devices breaking physical barriers in time and space, can be considered a consistent improvement.\\
Especially for companies working with digital healthcare data, it is important to underline the real possibility to provide each one softwares, even the legacy ones, with ad hoc modules to map their data into modern standards and EHR repository compliant formats, without radically changing their current workflow.
The adoption of standard communication architecture is at the same way very useful to enable interactions with the widest possible set of external devices, decreasing the time needed to complete specific tasks.\\ \\
The development of this prototype is valuable not only for Cardioline as a company but also as an experiment to break the current digital healthcare barriers.\\
Concluding, the economic and time effort required for this kind of evolution, has to be as lowest as possible, in particular dealing with e-health which is already strongly limited by countries and organizations burocracy. Hence the usage of cloud computing environment due to their optimized resource consumption and cost brings a great technological and business advantage.

\section{Future Develompent}
The initial request to keep the system portable over different cloud providers minimizing the vendor lock-in has been, in part, satisfied. Although each of the web services and web applications developed can be deployed on a Windows machine properly configured, the use of static storage cloud services as Amazon S3 is heavily integrated. Surely the persistency layer provided by Amazon RDS is completely portable towards wahever 
Once countries EHR will be opened to third-party stakeholders, the prototype developed in Cardioline will be immediately able to send its own produced documents through an appropriate communication channel.
Cardioline is still focused on improving the developed prototype. To respect the future European Law about privacy and sensible data it will be necessary to implement a strong-authentication mechanism and enable cardiologists to sign digitally reviewed exams.\\ \\
Furthermore a deep testing has to be done in order to analyze the system behaviour in case of huge traffic and resources demand.\\ \\
