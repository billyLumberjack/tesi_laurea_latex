%% -1	\part{part}
%% 0	\chapter{chapter}
%% 1	\section{section}
%% 2	\subsection{subsection}
%% 3	\subsubsection{subsubsection}
%% 4	\paragraph{paragraph}
%% 5	\subparagraph{subparagraph}
%% \part and \chapter are only available in report and book document classes

\chapter{Introduction}
ciccio~\label{cha:manna}
reference~\ref{cha:manna}

\section{State Of The Art}
\subsection{Ecg Solutions}
\subsubsection{Clinically-used 12-lead ECG}
12-Lead Ecg is on of the most used methodology in clinical cardiac medicine, it consists on attaching 10 electrodes on the surface of the limbs and the chest and therefore generates 12-lead: 12 groups of signals.\\
Traditionally each instrument generates vendor-specific waveform data formats which are compressed or encrypted by their algorithms.\cite{Hsieh2012}
\subsubsection{ECG Digitalization}
In 1989-1990 European, American and Japanese manufacturers jointly worked and agreed on the SCP-ECG development, to allow full interoperability between ECG devices and generic host systems.\\
Later in 1993 it became en European ENV (Europäische Norm Vorübergehend: a provisional eurocode to indicate they are within a test period).\\
The standard addresses specifically compression and communication of resting ECGs for different methodologies such as Holter, Exercise ECGs and real-time monitoring.\\
In 2002, the first open source online platform for SCP-ECG certification and conformance testing has been developed. Furthermore it allowed the visualization of the format which was the main limited deployment reason.\\

\cite{Chronaki} 
\subsection{Cloud Computing Application To Medical Device}


\section{Thesis Structure}
\label{sec:greetings}

Hello!
I greeted in section~\ref{sec:greetings}.