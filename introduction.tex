%% -1	\part{part}
%% 0	\chapter{chapter}
%% 1	\section{section}
%% 2	\subsection{subsection}
%% 3	\subsubsection{subsubsection}
%% 4	\paragraph{paragraph}
%% 5	\subparagraph{subparagraph}
%% \part and \chapter are only available in report and book document classes

\chapter{Introduction}
ciccio~\label{cha:manna}
reference~\ref{cha:manna}

\section{State Of The Art}
\subsection{Ecg Solutions}
\subsubsection{12-lead ECG}
12-Lead Ecg is on of the most used methodology in clinical cardiac medicine, it consists on attaching 10 electrodes on the surface of the limbs and the chest and therefore generates 12-lead: 12 groups of signals. The overall magnitude of the heart's electrical potential is then measured from twelve different angles (called leads) and is recorded in seconds over a period of time.\\
The graph of voltage versus time is referred to as electrocardiogram, which is considered the main tool for standard ECG analysis. \cite{electrocardiography_it} \cite{electrocardiography_en}\\
This methodology is fundamental for Myocardial infarction, Ischemia, Heart arrhythmia, Heart failure, Artificial cardiac pacemaker interested patients.
Traditionally each instrument generates vendor-specific waveform data formats which are compressed or encrypted by their own algorithms.\\
Unfortunately clinically used 12-lead ECG instruments are limited to the use in the hospitals. Due to the great variability of 12-lead ECG instruments and medical specialists interpretation skills, it remains a challenge to deliver rapid and accurate 12-lead ECG reports with senior cardioligists decision making support.\cite{Hsieh2012}
%\subsubsection{Dynamic ECG}
%\subsubsection{Holter monitor}
  


\subsubsection{ECG Digitalization}
The transmission, storage, and management of digital ECG signals have turned into major topics of debate and investigation. Within the context of this debate, the standardization of these processes has been a key issue lasting recent decades. \cite{trigo} The result is a multitude of format and standard which of course hinder the design and development of end-to-end standard-based systems.\\
A lot of mapping procedures between ECG formats has been created to solve the problem.\\
Indeed a wide variety of experience in the use of the SCP-ECG standard can be found in the literature over last two decades [73]–[81]. This proliferation of literature proves the interest of the scientific community in this standard, and makes SCP-ECG one of the most widespread initiatives in medical informatics standardization [82].
The history of the standard starts in 1989-1990 when European, American and Japanese manufacturers jointly worked and agreed on SCP-ECG development, to allow full interoperability between ECG devices and generic host systems.\\
The aim was the digital exchange of ECG signals and related metadata.
Later in 1993 it became an European ENV code (Europäische Norm Vorübergehend: a provisional eurocode to indicate a test period). But became an offical EN eurocode only in 2005, after two years in 2007 it has been revised and later in 2009 adopted as an ISO (ISO 11073-91064:2009).\\
An SCP-ECG revision Project Team lead by Paul Rubel and Alois Schloegl has been nominated in 2014, with the mission of keeping the useful and up-to-date parts of the standard more or less intact, removing or revising the outdated parts and extending the standard by including new measurements and recording modalities, such as long-term and stress test recordings.\cite{danilopani}\\
The standard addresses specifically compression and communication of resting ECGs for different methodologies such as Holter, Exercise ECGs and real-time monitoring.\\
In 2002, the first open source online platform for SCP-ECG certification and conformance testing has been developed.\cite{Chronaki}\\

\subsection{eHealth (Cloud Computing Application To Medical Device)}
E-Health is at the base of cloud computing for clinical use, which John Mitchell from Sydney university define as:"\textit{The health industry's equivalent of e-commerce}". From an integration perspective, e-health are:"\textit{integrated healthcare systems}" whose sum is much more than the one obtained from the single components. Even if the meaning varies among different contexts \cite{Eysenbach} it is possibile to define e-health as the set of technological themes in health today, whose initiatives do not originate necessarily with the patient. \cite{oh}\cite{DellaMea}
A technological improvement is represented by the e-health cloud computing birth, according to Foster et al. cloud computing is \textit{"A computing paradigm which is a pool of abstracted, virtualized, dynamically scalable, managing, computing power storage platforms and services  for on demand delivery over the Internet"} \cite{foster}, moreover CC offers a pay-as-you-use model helping healthcare industry cope with current and future demands, keeping the cost minimum. \cite{AbuKhousa}
Already in 2012 was possible to observe many healthcare providers shifting the burden of managing and mantaining complex health information technologies (HIT) to the Cloud Service Providers. \cite{foster}
Infact the adoption of CC technologies can significantly reduce IT costs while improving patient care services, solve resources scarcity problems due to distance and experts avaialability. \cite{AbuKhousa}
Looking at the landscape of the latest e-health cloud research prototypes,\\
Shah J.Miah et al. developed an e-health consultancy system utilizing cloud computing that enables doctors and healthcare workers to identify and treat non-communicable diseases in rural and remote communities in Bangladesh, a developing nation. \cite{MIAH2017311}\\\\
Sri Vijay BharatPeddi et al. propose a cloud-based mobile e-health daily calorie intake measurement system that can classify food objects in the plate and further compute the overall calorie of each food object with high accuracy. The novelty of the system is to avoid the  heavy offload of computational system functions to the cloud, but also employing an intelligent cloud-broker mechanism to strategically and efficiently utilize cloud instances \cite{PEDDI201771}\\\\
Mohamed Estai et al. showed a store-and-forward telemedicine platform “Remote-I” to assist in the screening of oral diseases using an image acquisition Android app operated by 17 teledental assistants. A total of 485 images (five images per case) were directly transmitted from the Android app to the server. Then a panel of five dental practitioners assessed the images and reported their diagnosis.\cite{mohamedestai}


\section{Thesis Structure}
\label{sec:greetings}

Hello!
I greeted in section~\ref{sec:greetings}.