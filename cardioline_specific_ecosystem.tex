\chapter{Cardioline Specific ecosystem}
\label{chapter:cardioline_specific_ecosystem}
Cardioline is present on the market from the early 60s, they started developing and producing innovative machines to record the heart acivity. Nowadays their thecnology improved, making their products producing documents in a digital format.
Thus the company started to sell an ecosystem composed by its products. Currently they provide heart rate monitor machines which produce digital documents, software tools to forward them to management system and client application or web based application to carry out the exam lifecycle.

\section{Software and Frameworks involved}
Because Cardioline SCP-ECG libraries are microsoft technologies based, their software are based on .NET platform.
The current Cardioline management software is a 3-tier MVC web application written in C# and deployed on windows server running machines.

\section{Project target And Requirements}
The scope is to provide future customers with additional features with respect to the current offered Ecosystem, avoiding the burden to keep up and running their own machines and infrastructure while Respecting the European jurisprudence in privacy subject and integrating their system with the national e-healthcare system. The company would like to supply their customer physicians and technicians with an electrocardiograph in order to make the perform cardio tests on their patients.\\
Later the documents should be uploaded to a customer dedicated cloud repository and an expert enabled to report it. Finally the user who submitted the exam can access the application and view the physician conclusions.
To carry out the target, these were the main challenges:
\begin{itemize}
    \item High network and computing load
    \item New security policies for data
    \item Adoption of the international HL7 standard
    \item Continuity with legacy software
\end{itemize}
The resulting prototype had to be sufficiently flexible and configurable to be adapted over several environments.

\paragraph{Functional and Non functional requirements}
As reported above, Cardioline together with FBK came up with additional features which the current version of their system does not support.
The following are the functional requirements.
\subparagraph{Anamnesis}
The future management system should let the single tenant decide wether or not to allow cardiologist submit their report even if the patient doctor didn't upload the exam with an anamnesis description. This feature is in particular important for pharmacies, because their technicians are not qualified to insert patient anamnesis. In this way every Cardioline customer (which correspond to hospital, clinic, etc..) can exert its own policy about anamnesis and reviews.
\subparagraph{Notifications}
Currently Cardioline web application software does not provide a notification features, mainly because it is hosted on LANs.
Because the architectural change enabled ubiquitus examination and analysis, it also imposes the development of a reliable notification system. Once a new electrocardiogram is submitted, if the system is in this way configured, a notification is sent to users which have anamnesi providing rights on it. If the option is not activated by the tenant user with exam review proviledges are notified.\\
In the case an anamnesis is submitted, the associated physician are notified for a new available ECG.\\
Finally when an ECG review is submitted by physician, the user who is taking care of the associated patient receives a notification of the process completion.

\paragraph{Proposed solutions}


\section{Provider choice}
It has been decided to face the computing and network load problem using Computing Cloud technologies, since they let the developer choose the control degree and customization over the system and how to structure the network.
In particular Cloud Computing providers offers this kinds of services:
\begin{itemize}
    \item Saas
    \item Paas
    \item Iaas
\end{itemize}
\paragraph{Amazon Elestic Beanstalk}
\label{paragraph:amazon_elastic_beanstalk}
[TO DO]
And different kind of network topology such as:
\begin{itemize}
    \item Private cloud
    \item Public cloud
\end{itemize}

%\section{System Architecture}