%% 2. 
%% (a)  (b) Implementation
%% (c) Workflow
%% (d) Long-term preservation

\chapter{Electronic health record}
Medical record, medical chart and health record are different terms to describe the documentation of a patient history and care.\\
Traditionally, health records were written on paper, maintained in folders divided into sections based on the type of note, and only one copy was available.
Computer Technology developed in the 60s and 70s laid the foundation for the development of the Electronic Health Record.\\
As inadequacies of the paper record became increasingly more apparent in 1992, the American Institute of Medicine advocated a shift from a paper-based to an electronic medical record one.\cite{Evans2016ElectronicHR}\\
An Electronic Health Record (EHR) is defined as a repository of many Electronic Medical Records (EMR), which are a digital version of a chart with patient information stored in a computer, it refers to everything possible to find in a paper chart, such as medical history, diagnoses, medications, immunization dates and allergies.\cite{emr}\\
Indeed represents a medical record within a single facility, such as a doctor's office or a clinic.\\
Around 1992 in US, EHRs were developed and used at a number of academic inpatient and outpatient medical facilities, but none contained all the information in the paper chart and most EHRs today are still a hybrid collection of computerized and paper data.
\cite{Evans2016ElectronicHR}\\
Because the regionalization of the Italian Healthcare System, there exists discrepancies among the different areas of the country, which the use of IT technologies may be useful to solve. In 2010 the Italian government introduced the electronic health record (EHR) implementation, which includes a minimum core of essential documents that should be created and updated by general practitioners.\\
During March 2017 the Italian e-health experts met for an \textit{InnovationLab}, there were government agencies, CEO and managers of important health companies, national care administrators and scholars. Their aim was to unlock the Italian health welfare because the state inadequacy, completing rapidly the e-health plan and the EHR (aka Fascicolo Sanitario Elettronico). They ended up with the \textit{"Carta di Salerno"}: an agreement to project the Italian Smart Health.
In 2002, an experts group from Cup2000 Spa designed a regional e-health network called SOLE (Sanità OnLinE) usefull for the FSE.
\cite{smarthealth}
\section{Accredited organization}
Health Level Seven International (HL7), is a non-profit organization involved in the development of international healthcare informatics interoperability standards known as Health Level 7 (HL7).\\
HL7 is an international community of healthcare subject matter experts and information scientists collaborating to create a framework (and related standards) for the exchange, integration, sharing, and retrieval of electronic health information.[2] HL7 promotes the use of such informatics standards within and among healthcare organizations to increase the effectiveness and efficiency of healthcare information delivery for the benefit of all.
\cite{hl7}
\section{Implementation}
\section{Workflow}
\section{Long-term preservation}